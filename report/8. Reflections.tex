\section{Reflections}

\subsection{Correctness of our model}
\subsection{Realism of the algorithm}
\subsection{Relation between expected behaviour and model results}
In general, our results are very alike those presented in the algorithm article \parencite{AlgorithmPaper}. We find similar correlations between the various variables and the iterations needed for the network to converge and consent on a singular decision. In some cases, we have been unable to replicate the exact experiment setup that was used in the original work, but even with the approximate setups we get similar results. In terms of our fourth research question, we conclude that the numbers match exceedingly well. For both network dependency values, network size, options and external interference, the results from our synthesized automata model closely conform to the proposed values from the original article.
While network dependency value was found to correlate with the amount of iterations needed to finish a decision process, we have also found that it is not a perfect metric. Often during random tests and setup, we would encounter somewhat large variance in the iteration counts, even if two networks had the same network dependency value. This suggests that there are other factors of network topology that could be very relevant for how the algorithm performs, and certain structures that may cause long, unintended processes that impede and slow down the decision making process. We have not been able to pin down any other specific, significant structure, but we theorize that they exist, and leave the discovery to possible future work.
